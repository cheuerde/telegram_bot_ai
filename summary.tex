\documentclass[12pt]{article}
\usepackage[utf8]{inputenc}
\usepackage{amsmath}
\usepackage{amsfonts}
\usepackage{amssymb}
\usepackage{graphicx}
\begin{document}
\section*{Number of Chapters: 6}
\subsection*{Overall Summary}


In the 1830s, rulers and nobility began leaving the Hebrides and the MacKenzie of Seaforth line died with the passing of Francis Mackenzie in 1781. In 2013, the second phase of the Museum of the Isles was completed, focused on the history of Lewis and opened in July 2016. The park is home to various species of trees, including beech, ash, elm, pine, elder, oak, horse chestnut and fir. Mary Perceval built a wall at Cuddy Point as a memento of her love and gratitude to James Matheson, who married in Edinburgh in November 1843 and died in Mentone, France in December 1878. The park is also home to rhododendrons, with many of the trees being destroyed by a storm in 1989
\newpage\section*{Chapter Summaries}
\subsection*{Chapter {1} Summary}
\begin{itemize}
 schließlich 1778 erneut eingesetzt. Er starb allerdings bereits 1781, wodurch die Herrschaft der MacKenzies über Lewis ein jähes Ende fand.[11] Matheson, Lever und Lews CastleBearbeiten James Matheson, 1844 In den 1830er Jahren begannen Herrscher und Adlige, die Hebriden zu verlassen und\end{itemize}
\subsection*{Chapter {2} Summary}
\begin{itemize}
 Jahr 2014 war das Projekt 2016 abgeschlossen und auf dem Gelände konnte das Museum of the Isles eröffnet werden. Eine weitere Förderung von 2,2 Mio. £ ermöglichte die Eröffnung des Hotels im November 2016.

\item  The MacKenzies of Seaforth line died with the passing of Francis Mackenzie\end{itemize}
\subsection*{Chapter {3} Summary}
\begin{itemize}
 17. November 1796/Verheiratet in Edinburgh, 9. November 1843/Gestorben in Mentone, Frankreich, 31. Dezember 1878.“[15][27]

\item  Im Jahr 2013 wurde die zweite Bauphase des Museums abgeschlossen, das sich auf die Geschichte von Lewis fokussiert und im Juli 2016 seine P\end{itemize}
\subsection*{Chapter {4} Summary}
\begin{itemize}
 Matheson of Lews in memory of my beloved father Roderick Matheson of Lemreway who died on the 17th day of November 1796/Married in Edinburgh 9th November 1843/Died in Mentone, France, 31st December 1878.

\item  17. November 1796: Roderick Matheson of Lemreway gestorben 
\item  9. November 1843: Roderick Matheson in Edinburgh verheirat\end{itemize}
\subsection*{Chapter {5} Summary}
\begin{itemize}
 bilden Kiefern und Laubbäume, während immergrüne Arten wie Chilenische Araukarie (Araucaria araucana) und Chinesische Fichte (Picea wilsonii) die Parkanlage im Winter zusätzlich schmücken.
\item  Mary Perceval built a wall at Cuddy Point as a memento of her love and gratitude to\end{itemize}
\subsection*{Chapter {6} Summary}
\begin{itemize}
 Mobile Ansicht

\item  Bilden Bergahorn, Buchen, Eschen, Ulmen, Kiefern, Ebereschen, Eichen, Rosskastanien und Tannen 
\item  Zahlreiche Exemplare fielen im Jahre 1989 einem Sturm zum Opfer
\item  Unter dem schützenden Schirm der Bäume gedeihen Rhododendren, insbesondere der.\end{itemize}
\end{document}